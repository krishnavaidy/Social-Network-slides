%%A Presentation by Krishna Vaidyanathan

\documentclass[aspectratio=169]{beamer}

\usepackage{hyperref}
\usepackage{color}
\usepackage{multirow}

%% Smart underlining -- from cdi-macros.tex
\def\ul#1{$\underline{\smash{\hbox{#1}}}$}

%% Shortcuts

\newcommand{\tabitem}{~~\llap{\textbullet}~~}

\newcommand{\bd}{\begin{description}}
\newcommand{\ed}{\end{description}}

\mode<presentation>
{
  \usetheme{Madrid}
  \useinnertheme{circles}
  \usecolortheme{beaver}
}
\usepackage[english]{babel}

\usepackage{times}
\usepackage[T1]{fontenc}
%
\title[Game-Theoretic Models]{Game-Theoretic Models of Information Overload in Social Networks}
\subtitle{A Presentation for CS886}
\author[Presented by Krishna Vaidyanathan]{Christian Borgs,\\Jennifer
    Chayes,\\Brian Karrer,\\Brendan Meeder,\\R. Ravi,\\Ray Reagans,\\Amin
    Sayedi\\\vspace{1em}Presented by Krishna Vaidyanathan}


\newcommand{\bi}{\begin{itemize}}
\newcommand{\ei}{\end{itemize}}

\newcommand{\bn}{\begin{enumerate}}
\newcommand{\en}{\end{enumerate}}

\begin{document}
\institute[UW]{University of Waterloo}
\frame[plain]{\titlepage}

\frame[plain]{\tableofcontents}

\section{Introduction}
\begin{frame}{Background}
    \bi
\item Increasing importance of social media.
    \pause
\item Some surveys claim the average person has five social media accounts and
    spends 1hr 40 mins on them every day \cite{telegraph15}.
    \pause
\item Increasing irrelevant updates on social media newsfeeds, or information overload.
    \ei
\end{frame}

\begin{frame}{Types of social networks}
    \bi
\item The paper considers two types of social media, namely:
    \pause
\item Symmetric: requires consent from both sides to maintain tie - eg.,
    Facebook.
    \pause
\item Asymmetric: requires consent from only one side to maintain tie - eg.,
    Twitter.
    \pause
\item Authors mainly look at asymmetric social networks.
    \ei
\end{frame}

\begin{frame}{Importance of information overload}
    \bi
\item Social networks make it convenient to get updates asynchronously.
    \pause
\item Makeup of newsfeed becomes very important to user.
    \pause
\item Mix of newsfeed is determined by the activity level of user's friends.
    \pause
\item \textit{How much one hears from one particular friend is not in user's 
        control.}
    \ei
\end{frame}

\begin{frame}{Models for social networks}
    \bi
\item  Assumptions of model:
    \pause
    \bi
\item Rate of sending updates is key decision variable.
    \pause
\item Updates from friends are useful, but excessive updates have diminishing
    value.
    \pause
\item Users can be partitioned as producers and consumers of information (80 -
    20 rule).
    \ei
    \ei
\end{frame}

\begin{frame}{Models for social networks}
    \bi
\item Followership: Users in network will stay in network but unfollow agents
    who give too frequent updates.
    \pause
\item Engagement: Users get frustrated by high update rate of followees and
    leave the social network.
    \ei
\end{frame}

\begin{frame}{Graph Model}
    \bi
\item Complete bipartite graph on two disjoint sets of nodes: producers (C), and
    consumers (F).
    \pause
\item Edge between producer $i$ and consumer $j$ is associated with a non-negative
    quality score $q_{ij}$.
    \pause
    \bi
\item $q_{ij}$ denotes utility consumer $j$ derives from producer $i$'s updates.
    \ei
    \pause
\item Producer $i$ updates at a frequency (rate) of $r_{i}$.
    \pause
\item Payoff for producer $i$ is $r_i$ times the number of followers he/she has.
    \ei
\end{frame}

\section{Followership model}
\begin{frame}
\end{frame}

\begin{frame}{References}
\begin{thebibliography}{9}
    \bibitem{borgs10}Borgs, C., Chayes, J., Karrer, B., Meeder, B., Ravi, R., Reagans,
    R., \& Sayedi, A. (2010). Game-theoretic models of information overload in
    social networks. In Algorithms and Models for the Web-Graph (pp. 146-161).
    Springer Berlin Heidelberg.
    \bibitem{telegraph15}
    \url{http://www.telegraph.co.uk/finance/newsbysector/mediatechnologyandtelecoms/11610959/Is-your-daily-social-media-usage-higher-than-average.html}
\end{thebibliography}

\end{frame}
\end{document}
